\documentclass[12pt,a4paper]{article}

% Packages essentiels
\usepackage[utf8]{inputenc}
\usepackage[T1]{fontenc}
\usepackage[french]{babel}
\usepackage{geometry}
\usepackage{graphicx}
\usepackage{hyperref}
\usepackage{fancyhdr}
\usepackage{tocloft}

% Configuration du chemin des images
\graphicspath{{figures/}{../figures/}}

% Configuration de la page
\geometry{margin=2cm, headheight=70pt, headsep=0pt, footskip=30pt}
\pagestyle{fancy}
\fancyhf{}
\fancyhead[L]{
    \includegraphics[height=60pt]{companyLogo.jpg}%
}
\fancyhead[C]{\textbf{ProTerra -- Spécification}}
\fancyhead[R]{\thepage}
\fancyfoot[C]{©2024 Droits réservés, Tremblay}
\renewcommand{\headrulewidth}{0.4pt}

% Appliquer le style fancy même aux pages spéciales
\fancypagestyle{plain}{
    \fancyhf{}
    \fancyhead[L]{
        \includegraphics[height=60pt]{companyLogo.jpg}%
    }
    \fancyhead[C]{\textbf{ProTerra -- Spécification}}
    \fancyhead[R]{\thepage}
    \fancyfoot[C]{©2025 Droits réservés, Tremblay, Le Gall}
    \renewcommand{\headrulewidth}{0.4pt}
}

% Informations du document
\title{ProTerra\\Spécification Technique}
\author{Matthieu}
\date{\today}

\begin{document}

\maketitle
\newpage

\vspace*{5cm} % Espace supplémentaire avant la table des matières
\tableofcontents
\newpage
\hspace{2cm}
\section{Introduction}
Ce document présente la spécification technique du projet ProTerra.

% TODO: Inclure les sections une fois créées
% \input{sections/1.Introduction}
% \input{sections/2.DescriptionGenerale}
% \input{sections/3.ExigencesSpecifiques}
% \input{sections/4.Signature}

\section{Description Générale}
Section à développer...

\section{Exigences Spécifiques}
Section à développer...

\section{Signature}
Section à développer...

\end{document}
